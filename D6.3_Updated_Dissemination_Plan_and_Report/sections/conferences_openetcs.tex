\section{Events Organised by openETCS Project Partners}\label{sct:openetcs_events}

%\begin{activity}{Workshop and Training Class on Formal Methods}
%	\entry{Date}{15-17/04/2013}
%	\entry{Location}{Munich (Germany)}
%	%\entry{Interested Partners}{DB, ERTMS Solutions, TWT, University of Rostock}
%	\desc{The project partners will organize a workshop on formal methods at the site of Deutsche Bahn in %Munich, Germany. Goal of the workshop is to give an overview and short introduction to the vast number of %formal methods which can potentially be used for the formalization process of the project. For this, the %format of an un-conference will be used. In a closing workshop we want to set up an evaluation matrix with %all relevant parameters. Moreover, there will be a training class on FORMALSPEC.}
%\end{activity}

%\begin{activity}{Special Session on Ensuring Safety in Industrial Critical Systems at IEEE INDIN 2013}
%	\entry{Date}{29-31/07/2013}
%	\entry{Location}{Bochum (Germany)}
%	%\entry{Interested Partners}{DB, ERTMS Solutions, TWT, University of Rostock}
%	\desc{TWT, Deutsche Bahn, the University of Rostock, and the project-external partner Infineon will organize a special session ``Ensuring Safety in Industrial Critical Systems'' (ESICS) at the IEEE International Conference on Industrial Informatics (INDIN). The purpose of the workshop is to leverage the discussion with respect to development methods of embedded safety-critical systems in transportation, thus building the foundation for an ecosystem of stakeholders and for dissemination of project results. Several contributions from the openETCS project are expected.}
%\end{activity}

%\begin{activity}{Test4Rail Symposium}
%	\entry{Date}{Q4/2013}
%	\entry{Location}{Braunschweig}
%	\entry{Organisator}{DLR}
%	\desc{It is the objective of the Symposium, to identify new approaches for the optimisation of tests of safety-critical, software-based railway systems. This objective shall be reached by exchange between science, industry and operators. The complexity of safety-critical and software-based systems in the railway sector as well as specific requirements related to safety and availability lead today to enormous effort for test and commissioning. On the other side modern methods and procedures for system tests are available from information science. }
%\end{activity}


%\begin{activity}{openETCS Conference 2013}
%	\entry{Date}{Q4 2013}
%	%\entry{Participants}{openETCS consortium}
%	\desc{Approximately by the End of 2013 a mid-term conference under the patronage of the UIC (Union Internationale des Chemins de Fer, engl.: International Union of Railways) in Paris is planned. It will be attracting potential users, which are railway companies world wide.}
%\end{activity}

\begin{activity}{openETCS Conference 2015}
	\entry{Date}{2015}
	%\entry{Participants}{openETCS consortium}
	\desc{After finishing the project in 2015 a second conference on the openETCS subject is planned to present the final results and most likely first marketable products at an adjacent product exhibition.}
\end{activity}

\begin{activity}{tbc}
	\entry{Date}{2015}
	%\entry{Participants}{openETCS consortium}
	\desc{DESCRIPTION}
\end{activity}