\section{Planned Future Dissemination Activities}\label{sct:planned_activities}

\subsection{Exhibitions and Trade Shows}

%\begin{activity}{Rencontre du numérique de l'ANR}
%	\entry{Frequency}{annual}
%	\entry{Next Event}{April 2013 (Paris, France)}
%	%\entry{Interested Partners}{CEA LIST}
%	\desc{Presentation of (French) projects related to openETCS; CEA LIST plans to present the Frama-C t%ool.}
%\end{activity}

\begin{activity}{ITEA \& ARTEMIS Co-summit 2015}
	\entry{Date}{10-11/3/2015}
	\entry{Location}{Berlin, Germany}
	%\entry{Interested Partners}{Alstom, Deutsche Bahn, ERSA, ERTMS-Solutions, General Electric, Siemens, SNCF}
	\desc{The ITEA \& ARTEMIS Co-summit will be used as platform to present the project, report about the current status of the openETCS project and present intermediate results. It is a great opportunity for exchanging ideas with the community and to disseminate and advertise the project as the event covers participants from industry, academia, public authorities and press from all over Europe.}
\end{activity}

\begin{activity}{Embedded World Exhibition \& Conference}
	\entry{Frequency}{annual}
	\entry{Next Event}{February 2015 (Nuremberg, Germany)}
	\website{http://www.embedded-world.eu/}
	%\entry{Interested Partners}{Fraunhofer ESK}
	\desc{The annual EmbeddedWorld Exhibition \& Conference is the world largest trade exhibition in the area of embedded systems. It is planned to present openETCS partner results at the exhibition as well as at the conference.}
\end{activity}


\subsection{Conferences and Workshops}


%\begin{activity}{41th Symposium ``Moderne Schienenfahrzeuge''}
%	\entry{Type}	{Talk}
%	\entry{Title}	{Open ETCS: Ein internationales ITEA2-Projekt begleitet den Wandel}
%	\entry{Speaker}	{Klaus-R\"{u}diger Hase (DB)}
%	\entry{Date}		{7-10/04/2013}
%	\entry{Location}	{Graz, Austria}
%	\entry{Website}		{\url{http://www.schienenfahrzeugtagung.at/}}
%	\desc{The symposium focuses on highspeed railway traffic, railway transport vehicles, rail fraight transport, vehicle components, and interaction of wheel and track. The platform of the symposium will be used to present the openETCS approach to a large audience of distinguished experts in railway business.}
%\end{activity}

\begin{activity}{SIGNAL+DRAHT-Congress}
	\entry{Frequency}{annual}
	\website{http://www.eurailpress.de/events/eurailpress-events.html}
	\desc{A talk about the openETCS project has already been given at the 12th international SIGNAL+DRAHT congress in 2012 (see Section~\ref{sct:past_activities}). A follow up talk is planned for 2014 or 2015 in order to present the results of the project.}
\end{activity}

\begin{activity}{IEEE Vehicular Technology Conference}
	\entry{Frequency}{bi-annual}
	\entry{Next Events}{May 2015 (Glasgow, Scotland)}
	\website{http://www.ieeevtc.org/vtc2015spring}
	%\entry{Interested Partners}{\red{@all: check whether this is relevant for you!}}
	\desc{This semi-annual flagship conference of the IEEE Vehicular Technology Society will bring together individuals from academia, government, and industry to discuss and exchange ideals in the fields of wireless, mobile, and vehicular technology. The conference will feature world-class plenary speakers, tutorials, and technical as well as application sessions.}
\end{activity}

\begin{activity}{IFIP International Conference on Testing Software and Systems (ICTSS)}
	\entry{Frequency}{annual}
	%\entry{Interested Partners}{CEA LIST}
	\desc{The well-established ICTSS series of international conferences addresses the conceptual, theoretic, and practical challenges of testing software systems, including communication protocols, services, distributed platforms, middleware, embedded systems, and security infrastructures.}
\end{activity}

\begin{activity}{IEEE International Conference on Industrial Informatics (INDIN'15)}
	\entry{Frequency}{annual}
	\entry{Next Events}{July 2015 (Cambridge, UK)}
	\website{http://www.ieeevtc.org/vtc2015spring}
	%\entry{Interested Partners}{\red{@all: check whether this is relevant for you!}}
	\desc{The purpose of the IEEE INDIN international conference is to provide a forum for presentation and discussion of the state-of-art and future perspectives of industrial information technologies. Industry experts, researchers and academics are gathering together to share ideas and experiences surrounding frontier technologies, breakthroughs, innovative solutions, research results, as well as initiatives related to industrial informatics and their applications.}
\end{activity}

\begin{activity}{Design, Automation, and Test in Europe (DATE)}
	\entry{Frequency}{annual}
	\entry{Next Events}{March 2015, Grenoble, France}
	\website{http://www.date-conference.com/}
	%\entry{Interested Partners}{\red{@all: check whether this is relevant for you!}}
	\desc{DATE is a leading international event and unique networking opportunity for design and engineering of Systems-on-Chip, Systems-on-Board and Embedded Systems Software.}
\end{activity}

\begin{activity}{IEEE Real-Time Systems Symposium (RTSS)}
	\entry{Frequency}{annual}
	\entry{Next Events}{December 2014, Rome, Italy}
	\website{http://www.wikicfp.com/cfp/servlet/event.showcfp?eventid=37166&copyownerid=619}
	%\entry{Interested Partners}{\red{@all: check whether this is relevant for you!}}
	\desc{The IEEE Real-Time Systems Symposium (RTSS) is the premier conference in the area of real-time systems, presenting innovations in the field with respect to theory and practice. RTSS provides a forum for the presentation of high-quality, original research covering all aspects of real-time systems design, analysis, implementation, evaluation, and experiences. RTSS’14 continues the trend of making RTSS an expansive 
and inclusive symposium, looking to embrace new and emerging areas of 
realtime systems research. }
\end{activity}

\begin{activity}{Real-Time and Embedded Technology and Applications Symposium (RTAS)}
	\entry{Frequency}{annual}
	\entry{Next Events}{April 2015, Seattle, Washington}
	\website{http://2015.rtas.org/}
	%\entry{Interested Partners}{\red{@all: check whether this is relevant for you!}}
	\desc{RTAS’15, the 21st in a series of annual conferences sponsored by the IEEE, will be held in Seattle, Washington, as part of the Cyber-Physical Systems Week (CPSWeek) in April, 2015. CPS Week 2015 will bring together leading conferences, including the International Conference on Information Processing in Sensor Networks (IPSN’15), the International Conference on Hybrid Systems (HSCC’15), the International Conference on Cyber-Physical Systems (ICCPS’15), the International Conference on High Confidence Networked Systems (HiCoNS’15) and RTAS’15.}
\end{activity}

\begin{activity}{IFIP International Conference on Testing Software and Systems (ICTSS)}
	\entry{Frequency}{annual}
	%\entry{Interested Partners}{CEA LIST}
	\desc{The well-established ICTSS series of international conferences addresses the conceptual, theoretic, and practical challenges of testing software systems, including communication protocols, services, distributed platforms, middleware, embedded systems, and security infrastructures.}
\end{activity}

\begin{activity}{Workshop Methoden und Beschreibungssprachen zur Modellierung und Verifikation von Schaltungen und Systemen (MBMV)}
	\entry{Frequency}{annual}
	\entry{Next Events}{March 2015}
	\website{https://www.tu-chemnitz.de/etit/sse/mbmv2015/}
	%\entry{Interested Partners}{\red{@all: check whether this is relevant for you!}}
	\desc{Der Workshop hat es sich zum Ziel gesetzt, neueste Trends, Ergebnisse und aktuelle Probleme auf dem Gebiet der Methoden zur Modellierung und Verifikation sowie der Beschreibungssprachen digitaler, analoger und Mixed-Signal-Schaltungen zu diskutieren. Er soll somit ein Forum zum Ideenaustausch sein.}
\end{activity}

\begin{activity}{ACM Symposium on Applied Computing / Software Verification and Testing Track (ACM SAC)}
	\entry{Frequency}{annual}
	\entry{Next Events}{April 2015, Salamanca Spain}
	\website{http://fmt.cs.utwente.nl/conferences/sac-svt2015/}
	%\entry{Interested Partners}{\red{@all: check whether this is relevant for you!}}
	\desc{The ACM Symposium on Applied Computing (SAC) has gathered scientists 
from different areas of computing over the past twenty-nine 
years. The forum represents an opportunity to interact with different 
communities sharing an interest in applied computing. }
\end{activity}

\begin{activity}{Nasa Formal Methods Symposium (NFM)}
	\entry{Frequency}{annual}
	\entry{Next Event}{April 2015, Pasadena, California, USA)}
	\website{http://www.NASAFormalMethods.org/nfm2015}
	%\entry{Interested Partners}{CEA LIST}
	\desc{The NASA Formal Methods Symposium is a forum for theoreticians and practitioners from academia, industry, and government, with the goals of identifying challenges and providing solutions to achieving assurance in mission- and safety-critical systems.}
\end{activity}

\begin{activity}{FORMS/FORMAT}
	\entry{Frequency}{bi-annual}
	\entry{Next Event}{2016 (Braunschweig, Germany)}
	\website{http://www.forms-format.de}
	%\entry{Interested Partners}{TU Braunschweig}
 	\desc{The symposium FORMS/FORMAT offers scientists facing formal techniques, practitioners and managers, developers and consultants of automotive and railway industries as well as traffic system operators with interest in formal methods an accepted platform for the exchange of scientific experience and the transfer of practical description means, methods and tools for complex automation systems. With its focus on formal methods and transportation it provides an ideal forum to present openETCS and its results.}
\end{activity}

\begin{activity}{International Conference on Software Engineering and Formal Methods (SEFM)}
	\entry{Frequency}{annual}
	\entry{Next Event}{Sep. 2015 York, UK}
	\website{http://www.cs.york.ac.uk/sefm2015/}
	%\entry{Interested Partners}{Institut Telecom}
	\desc{The SEFM conference series aims to advance the state of the art and usage of formal methods in the industry, being a great hub for discussion on formal methods and software engineering.}
\end{activity}

\begin{activity}{IEEE International Conference on Software Testing (ICST)}
	%\entry{Interested Partners}{Institut Telecom}
	\entry{Frequency}{annual}
	\entry{Next Event}{Sep. 2015 York, UK}
	\website{http://icst2015.ist.tu-graz.ac.at/}
	\desc{ICST is a conference series that specialises in Software Testing and Verification \& Validation. Topics discussed range from Verification \& Validation, testing, model checking, among others.}
\end{activity}

\subsection{Journals and Magazines}

\begin{activity}{Journal of Statistical Software}
	\website{http://www.jstatsoft.org/}
	%\entry{Interested Partners}{Institut Telecom}
	\desc{Specializing in statistical software and algorithms, it provides a medium to publish about different techniques used for the validation \& verification of embedded systems.}
\end{activity}


\begin{activity}{SIGNAL+DRAHT}
	\website{http://www.eurailpress.de/verlag/zeitschriften/signal-draht/profil.html}
	\desc{A talk about the openETCS project has already been given at the 12th international SIGNAL+DRAHT congress in 2012 (see \ref{sct:past_activities}). For 2014 or 2015 it is planned to publish an article about the project results in the accompanying SIGNAL+DRAHT magazine.}
\end{activity}

\begin{activity}{International Railway Journal (IRJ)}
	\website{http://www.railjournal.com}
	\desc{The journal was launched in 1960 and is the world's first globally-distributed magazine for the railway industry. It is written for senior managers and engineers of the world's railways and transit systems, ministers of transport, manufacturers, railway planners, and consultants. Concerning openETCS, it is planned to publish an article about the simulator which will be developed for the purpose of demonstration in the project.}
\end{activity}

\begin{activity}{ZEV Rail}
	\website{http://www.zevrail.de}
	\desc{ZEV Rail is the accompanying journal to the ``Moderne Schienenfahrzeuge'' symposium. An article with the title ``OpenETCS: Ein internationales ITEA2-Projekt begleitet den Wandel'' is currently under submission.}
	
\end{activity}

\subsection{openETCS-Specific Events}

\begin{activity}{IEEE INDIN 2015}
	\entry{Type}	{Workshop / Special Session}
	\entry{Date}		{22-25/7/2015}
	\entry{Location}	{Cambridge, UK}
	%\entry{Website}		{\url{http://www.indin2013.org/}}
	\desc{The ESICS workshop at INDIN 2013 was a success and thus it is planned to organise another workshop for INDIN 2015. It will again focus on critical systems, safety and security.}
\end{activity}

\begin{activity}{openETCS Conference/Final Workshop 2015}
	\entry{Date}{2015}
	%\entry{Participants}{openETCS consortium}
	\desc{At the end of the project in 2015 a second conference is planned to present the final results, possible follow-up activities and exploitation perspectives.}
\end{activity}


