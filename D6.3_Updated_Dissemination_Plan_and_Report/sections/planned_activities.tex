\section{Planned Future Dissemination Activities}\label{sct:planned_activities}

\subsection{Exhibitions and Trade Shows}

%\begin{activity}{Rencontre du numérique de l'ANR}
%	\entry{Frequency}{annual}
%	\entry{Next Event}{April 2013 (Paris, France)}
%	%\entry{Interested Partners}{CEA LIST}
%	\desc{Presentation of (French) projects related to openETCS; CEA LIST plans to present the Frama-C t%ool.}
%\end{activity}

\begin{activity}{ITEA \& ARTEMIS Co-summit 2015}
	\entry{Date}{10-11/3/2015}
	\entry{Location}{Berlin, Germany}
	%\entry{Interested Partners}{Alstom, Deutsche Bahn, ERSA, ERTMS-Solutions, General Electric, Siemens, SNCF}
	\desc{The ITEA \& ARTEMIS Co-summit will be used as platform to present the project, report about the current status of the openETCS project and present intermediate results. It is a great opportunity for exchanging ideas with the community and to disseminate and advertise the project as the event covers participants from industry, academia, public authorities and press from all over Europe.}
\end{activity}

\begin{activity}{Embedded World Exhibition \& Conference}
	\entry{Frequency}{annual}
	\entry{Next Event}{February 2015 (Nuremberg, Germany)}
	\website{http://www.embedded-world.eu/}
	%\entry{Interested Partners}{Fraunhofer ESK}
	\desc{The annual EmbeddedWorld Exhibition \& Conference is the world largest trade exhibition in the area of embedded systems. It is planned to present openETCS partner results at the exhibition as well as at the conference.}
\end{activity}


\subsection{Conferences and Workshops}


\begin{activity}{41th Symposium ``Moderne Schienenfahrzeuge''}
	\entry{Type}	{Talk}
	\entry{Title}	{Open ETCS: Ein internationales ITEA2-Projekt begleitet den Wandel}
	\entry{Speaker}	{Klaus-R\"{u}diger Hase (DB)}
	\entry{Date}		{7-10/04/2013}
	\entry{Location}	{Graz, Austria}
	\entry{Website}		{\url{http://www.schienenfahrzeugtagung.at/}}
	\desc{The symposium focuses on highspeed railway traffic, railway transport vehicles, rail fraight transport, vehicle components, and interaction of wheel and track. The platform of the symposium will be used to present the openETCS approach to a large audience of distinguished experts in railway business.}
\end{activity}

%\begin{activity}{ERTMS World Conference}
%	\entry{Frequency}{bi-annual}
%	\entry{Next Event}{2014 (Istanbul, Turkey)}
%	\website{http://www.ertms-conference2014.com}
%	\desc{As the ETCS standard is one part of the European Rail Traffic Management System (ERTMS) this conference is highly recommended to present openETCS.}
%\end{activity}

\begin{activity}{SIGNAL+DRAHT-Congress}
	\entry{Frequency}{annual}
	\website{http://www.eurailpress.de/events/eurailpress-events.html}
	\desc{A talk about the openETCS project has already been given at the 12th international SIGNAL+DRAHT congress in 2012 (see Section~\ref{sct:past_activities}). A follow up talk is planned for 2014 or 2015 in order to present the results of the project.}
\end{activity}

\begin{activity}{IEEE Vehicular Technology Conference}
	\entry{Frequency}{bi-annual}
	\entry{Next Events}{May 2015 (Glasgow, Scotland)}
	\website{http://www.ieeevtc.org/vtc2015spring}
	%\entry{Interested Partners}{\red{@all: check whether this is relevant for you!}}
	\desc{This semi-annual flagship conference of the IEEE Vehicular Technology Society will bring together individuals from academia, government, and industry to discuss and exchange ideals in the fields of wireless, mobile, and vehicular technology. The conference will feature world-class plenary speakers, tutorials, and technical as well as application sessions.}
\end{activity}

%\begin{activity}{International Conference on Design and Operation in Railway Engineering (Comprail)}
%	\entry{Frequency}{bi-annual}
%	\entry{Next Event}{2014}
%	%\entry{Interested Partners}{Fraunhofer ESK (Session: Computer techniques and simulations)}
%	\desc{The aim of the conference is to review the latest developments in railway engineering and update the use of advanced systems, promoting general awareness throughout the business management, design, manufacture and operation of railways.}
%\end{activity}

%==================================
% No input
%==================================
%\begin{activity}{International Railway Safety Conference (IRSC)}
%	\entry{Frequency}{bi-annual}
%	\entry{Next Event}{2014}
%	\desc{\red{@all: Please check if this is still relevant and complete it. Otherwise this entry will be removed.}}
%\end{activity}

%==================================
% No input
%==================================
%\begin{activity}{Conference on Railway Engineering (CORE)}
%	\entry{Frequency}{bi-annual}
%	\entry{Next Event}{2014}
%	\desc{\red{@all: Please check if this is still relevant and complete it. Otherwise this entry will be removed.}}
%\end{activity}

%==================================
% No input
%==================================
%\begin{activity}{Fachtagung Technische Zuverl\"{a}ssigkeit (TTZ)}
%	\entry{Frequency}{bi-annual}
%	\entry{Next Event}{April 2013 (Leonberg, Germany)}
%	\website{http://www.vdi-wissensforum.de/de/nc/angebot/detailseite/event/02TA502013}
%	\desc{\red{@all: Please check if this is still relevant and complete it. Otherwise this entry will be removed.}}
%\end{activity}

%\begin{activity}{World Congress on Railway Research 2013 (WCRR 2015)}
%	\entry{Frequency}{bi-annual}
%	\entry{Next Event}{Nov. 2013 (Sydney, Australia)}
%	\website{http://www.wcrr2013.org/}
%	%\entry{Interested Partners}{TU Braunschweig}
%	\desc{The World Congress on Railway Research is the world's foremost international forum for the promotion, development and exchange of the latest innovations in the global rail industry.Therefore it can be used to promote openETCS and our first results to researchers, manufacturers and operators from across the global rail industry. }
%\end{activity}

%\begin{activity}{IEEE International Conference on Intelligent Rail Transportation (ICIRT)}
%	\entry{Next Event}{Aug. 2013 (Beijing, China)}
%	\website{http://www.ieee-icirt.org/}
	%\entry{Interested Partners}{TU Braunschweig}
%	\desc{Since their is a high interest in openETCS and its results in Asia, the 2013 IEEE International Conference on Intelligent Rail Transportation provides a forum to communicated with potential partners. The theme of ICIRT 2013 is "Safe, Green \& Intelligent Rail".}
%\end{activity}


\begin{activity}{IFIP International Conference on Testing Software and Systems (ICTSS)}
	\entry{Frequency}{annual}
	%\entry{Interested Partners}{CEA LIST}
	\desc{The well-established ICTSS series of international conferences addresses the conceptual, theoretic, and practical challenges of testing software systems, including communication protocols, services, distributed platforms, middleware, embedded systems, and security infrastructures.}
\end{activity}

\begin{activity}{European Safety and Reliability Conference (ESREL)}
	\entry{Frequency}{annual}
%	\entry{Next Event}{Sept. 2013 (Amsterdam, Netherlands)}
	\desc{Safety, reliability and risk management become more and more important in an always more challenging and competitive environment, in every industry and human activity: multidisciplinary approaches to safety and reliability engineering and risk management become more and more necessary and attractive.

	This annual conference will provide a forum for presentation and discussion of scientific works covering theories and methods in the field of risk, safety and reliability, and their application to a wide range of industrial, civil and social sectors and problem areas. ESREL 2011 will also be an opportunity for researchers and practitioners, academics and engineers to meet, exchange ideas and gain insight from each other.}
\end{activity}

%==================================
% No input by ALL4TEC
%==================================
%\begin{activity}{$\boldsymbol{\lambda\mu}$ Symposium}
%	\entry{Frequency}{bi-annual}
%	\entry{Next Event}{2014 (France)}
%	\website{http://www.imdr.fr/en/sommaire/index.php}
%	%\entry{Interested Partners}{ALL4TEC}
%	\desc{French safety and reliability conference, \tbc[by ALL4TEC]}
%\end{activity}

\begin{activity}{Nasa Formal Methods Symposium (NFM)}
	\entry{Frequency}{annual}
	\entry{Next Event}{April 2015 (Pasadena, USA)}
	\website{http://nasaformalmethods.org/}
	%\entry{Interested Partners}{CEA LIST}
	\desc{The NASA Formal Methods Symposium is a forum for theoreticians and practitioners from academia, industry, and government, with the goals of identifying challenges and providing solutions to achieving assurance in mission- and safety-critical systems.}
\end{activity}

\begin{activity}{Integrated Formal Methods (iFM)}
	\entry{Frequency}{annual}
	\website{http://ifm2014.cs.unibo.it/}
	%\entry{Interested Partners}{CEA LIST}
	\desc{The iFM conference series seeks to further research into hybrid approaches to formal modeling and analysis; i.e., the combination of (formal and semi-formal) methods for system development, regarding modeling and analysis, and covering all aspects from language design through verification and analysis techniques to tools and their integration into software engineering practice.}
\end{activity}

\begin{activity}{FORMS/FORMAT 2015}
	\entry{Frequency}{bi-annual}
%	\entry{Next Event}{Jan. 2014 (Braunschweig, Germany)}
	\website{http://www.forms-format.de}
	%\entry{Interested Partners}{TU Braunschweig}
 	\desc{The symposium FORMS/FORMAT offers scientists facing formal techniques, practitioners and managers, developers and consultants of automotive and railway industries as well as traffic system operators with interest in formal methods an accepted platform for the exchange of scientific experience and the transfer of practical description means, methods and tools for complex automation systems. With its focus on formal methods and transportation it provides an ideal forum to present openETCS and its results.}
\end{activity}

\begin{activity}{International Conference on Software Engineering and Formal Methods (SEFM)}
	\entry{Frequency}{annual}
%	\entry{Next Event}{Sep. 2013 (Madrid, Spain)}
	\website{http://sefm2014.inria.fr/}
	%\entry{Interested Partners}{Institut Telecom}
	\desc{The SEFM conference series aims to advance the state of the art and usage of formal methods in the industry, being a great hub for discussion on formal methods and software engineering.}
\end{activity}

\begin{activity}{IEEE International Conference on Software Testing (ICST)}
	%\entry{Interested Partners}{Institut Telecom}
	\entry{Frequency}{annual}
	\website{http://www.icst.lu/}
	\desc{ICST is a conference series that specialises in Software Testing and Verification \& Validation. Topics discussed range from Verification \& Validation, testing, model checking, among others.}
\end{activity}


\begin{activity}{International Conference on Model Driven Engineering Languages \& Systems (MODELS)}
	\entry{Frequency}{annual}
	\entry{Next Event}{September/October 2015 (Ottawa, Canada)}
	\website{http://www.modelsconference.org/}
	%\entry{Interested Partners}{Fraunhofer ESK}
	\desc{The International Conference on Model Driven Engineering Languages \& Systems (Models) is the premier conference series for model-based software and system engineering. The conference covers all aspects of modeling, from languages and methods to tools and applications.}
\end{activity}

\begin{activity}{IEEE International Conference on Software Engineering (ICSE)}
	\entry{Frequency}{annual}
	\entry{Next Event}{April 2015 (Nanjing, Jiangsu, China)}
	\website{http://icse2015.org/}
	%\entry{Interested Partners}{Fraunhofer ESK, Institut Telecom}
	\desc{The International Conference on Software Engineering (ICSE) is the premier software engineering conference, providing a forum for researchers, practitioners and educators to present and discuss the most recent innovations, trends, experiences and concerns in the field of software engineering.}
\end{activity}

%\begin{activity}{BICCnet Innovation Forum Embedded Systems}
%	\entry{Frequency}{annual}
%	\entry{Next Event}{April 2013 (Munich, Germany)}
%	\website{http://bicc-net.de/termine/termin/innovation-forum-embedded-systems/}
	%\entry{Interested Partners}{Fraunhofer ESK}
%	\desc{The Bavarian Information and Communication Technology Cluster (BICCnet) organizes the annual Innovation Forum Embedded Systems (IFES). The IFES offers talks, panel discussions and a trade exhibition addressing cross domain relevant topics in the area of networked embedded systems. It is planned to present openETCS partner results in talks as well as at the trade exhibition.}
%\end{activity}

\begin{activity}{IEEE/ACM International Conference on Automated Software Engineering (ASE)}
	\entry{Frequency}{annual}
	\entry{Next Event}{Nov. 2015 (Lincoln, Nebraska, USA)}
	\website{http://ase-conferences.org/}
	%\entry{Interested Partners}{Institut Telecom}
	\desc{The ASE conference series is a great forum for automatic software engineering, with main research topics on techniques and tools for the automation of testing, analysing and maintenance of software systems.}
\end{activity}

%\begin{activity}{International Conference on Quality Software}
%	\entry{Frequency}{annual}
%	\entry{Next Event}{July 2013 (Nanjing, China)}	
%	\website{http://software.nju.edu.cn/qsic/index.html}
	%\entry{Interested Partners}{Institut Telecom}
%	\desc{The QSIC conference series aims to improve the quality assurance and overall verification \& validation techniques. Main topics discussed are software testing and quality, debugging and formal methods.}
%\end{activity}


\begin{activity}{ACM Symposium on Applied Computing (SAC)}
	\entry{Frequency}{annual}
	\entry{Next Event}{April 2015 (Salamanca, Spain)}
	\website{http://www.acm.org/conferences/sac/sac2015/}
	%\entry{Interested Partners}{CEA LIST, Institut Telecom}
	\desc{Over the past years, the ACM Symposium on Applied Computing (SAC), the primary SIGAPP Annual Conference, has become a primary forum for applied computer scientists, computer engineers, software engineers, and application developers from around the world to interact and present their work.}
\end{activity}


\begin{activity}{International Conference on Software Engineering and Knowledge Engineering (SEKE)}
	\entry{Frequency}{annual}
	\entry{Next Event}{November 2015 (Istanbul, Turkey)}
	\website{http://www.waset.org/conference/2015/11/istanbul/ICSEKE}
	%\entry{Interested Partners}{Institut Telecom}
	\desc{Specializing in software engineering, the SEKE conference series provides a great forum for discussion around this topic.}
\end{activity}


\begin{activity}{EclipseCon Europe}
	\entry{Frequency}{annual}
	\entry{Next Event}{Oct. 2014 (Ludwigsburg, Germany)}
	\website{http://www.eclipsecon.org}
	%\entry{Interested Partners}{EclipseSource}
	\desc{EclipseCon is one of the largest open source conferences of the world. Several consortia such as POLARSYS, the Automotive Group and the OPEES group are presenting their results and are looking for potential collaboration partners.}
\end{activity}

\subsection{Journals and Magazines}

\begin{activity}{Journal of Statistical Software}
	\website{http://www.jstatsoft.org/}
	%\entry{Interested Partners}{Institut Telecom}
	\desc{Specializing in statistical software and algorithms, it provides a medium to publish about different techniques used for the validation \& verification of embedded systems.}
\end{activity}


\begin{activity}{SIGNAL+DRAHT}
	\website{http://www.eurailpress.de/verlag/zeitschriften/signal-draht/profil.html}
	\desc{A talk about the openETCS project has already been given at the 12th international SIGNAL+DRAHT congress in 2012 (see \ref{sct:past_activities}). For 2014 or 2015 it is planned to publish an article about the project results in the accompanying SIGNAL+DRAHT magazine.}
\end{activity}

\begin{activity}{International Railway Journal (IRJ)}
	\website{http://www.railjournal.com}
	\desc{The journal was launched in 1960 and is the world's first globally-distributed magazine for the railway industry. It is written for senior managers and engineers of the world's railways and transit systems, ministers of transport, manufacturers, railway planners, and consultants. Concerning openETCS, it is planned to publish an article about the simulator which will be developed for the purpose of demonstration in the project.}
\end{activity}

%\begin{activity}{Eisenbahntechnische Rundschau (ETR)}
%	\website{http://www.eurailpress.de/verlag/zeitschriften/eisenbahntechnische-rundschau/profil.html}
%	\desc{\red{@all, especially DB: Check whether this is still relevant, then please complete it. Otherwise it will be removed.}}
%\end{activity}

\begin{activity}{ZEV Rail}
	\website{http://www.zevrail.de}
	\desc{ZEV Rail is the accompanying journal to the ``Moderne Schienenfahrzeuge'' symposium. An article with the title ``OpenETCS: Ein internationales ITEA2-Projekt begleitet den Wandel'' is currently under submission.}
	
\end{activity}

\subsection{openETCS-Specific Events}

\begin{activity}{IEEE INDIN 2015}
	\entry{Type}	{Workshop / Special Session}
	\entry{Date}		{22-25/7/2015}
	\entry{Location}	{Cambridge, UK}
	%\entry{Website}		{\url{http://www.indin2013.org/}}
	\desc{The ESICS workshop at INDIN 2013 was a success and thus it is planned to organise another workshop for INDIN 2015. It will again focus on critical systems, safety and security.}
\end{activity}

\begin{activity}{openETCS Conference/Final Workshop 2015}
	\entry{Date}{2015}
	%\entry{Participants}{openETCS consortium}
	\desc{At the end of the project in 2015 a second conference is planned to present the final results, possible follow-up activities and exploitation perspectives.}
\end{activity}


