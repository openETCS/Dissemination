\section{Past Project Dissemination Activities}\label{sct:past_activities}

\subsection{ITEA/ARTEMIS Summits}
\begin{activity}{ITEA \& ARTEMIS Co-summit 2012: Sharing a Vision for ICT Innovation}
	\entry{Date}{30-31/10/2012}
	\entry{Location}{Paris, France}
	%\entry{Interested Partners}{Alstom, Deutsche Bahn, ERSA, ERTMS-Solutions, General Electric, Siemens, SNCF}
	\desc{The ITEA \& ARTEMIS Co-summit 2012 was used to introduce the openETCS project to a wide audience ranging from industry to academia, public authorities and press from all over Europe.}
\end{activity}

\begin{activity}{ITEA \& ARTEMIS Co-summit 2013: Software innovation: boosting high-tech employment and industry}
	\entry{Date}{4-5/1/2013}
	\entry{Location}{Stockholm, Sweden}
	%\entry{Interested Partners}{Alstom, Deutsche Bahn, ERSA, ERTMS-Solutions, General Electric, Siemens, SNCF}
	\desc{The ITEA \& ARTEMIS Co-summit 2013 was used to present the intermediate openETCS project results to a wide audience ranging from industry to academia, public authorities and press from all over Europe.}
\end{activity}

%##########################################################################
%Journal contributions
%##########################################################################
\subsection{Journal contributions}
\begin{activity}{European Railway Review, Volume 18, Issue 3, 2012}
	\entry{Type}	{Journal Article}
	\entry{Title}	{openETCS: Applying `Open Proofs' to European Train Control}
	\entry{Pages} {30--34}
	\entry{Author}	{Klaus-R\"{u}diger Hase (DB)}
	\entry{Publisher} {Russell Publishing Ltd., Kent, UK}
	\entry{website} {\url{http://europeanrailwayreview.com/tag/openetcs}}
	\desc{European Railway Review is the leading bi-monthly technical journal for the European rail industry. Featuring articles and news about the latest technologies and developments, the magazine is essential reading for people involved in the railway business. Here, an overview article of the openETCS project, covering the technical background and the project goals, has been published by DB Netz AG.}
\end{activity}

\begin{activity}{Science of Computer Programming, Volume 91, Part B, 2014}
	\entry{Type}	{Journal Article}
	\entry{Title}	{Dependability in Open Proof Software with Hardware Virtualization – The Railway Control Systems Perspective}
	\entry{Pages} {188--215}
	\entry{Author}	{Johannes Feuser, Jan Peleska}
	\entry{Publisher} {Elsevier B.V., Amsterdam, Netherlands}
	\entry{website}	{\url{http://www.sciencedirect.com/science/article/pii/S0167642313002001}}
	\desc{Using the openETCS initiative as a starting point, we describe how open software can be applied in combination with platform-specific, potentially closed-source extensions, in the development, verification, validation and certification of safety-critical railway control systems. To achieve certification credit for safety-critical system developments, evidence about numerous development, verification and validation artifacts has to be provided. Our focus is therefore on open models, and a model-driven development approach ensures that a large portion of the artifacts is automatically generated from the model. This strategy is illustrated by means of the ETCS standard, as far as applicable to the ETCS on-board computer managing train control and train protection. We show that a domain-specific language is suitable to cover all modeling aspects for this computer, starting from the ETCS standard itself and ending at supplier-specific adaptations extending the re-usable core model in concrete developments. In order to re-use certification credits once achieved for the re-usable core model, we suggest virtualization of run-time environments, so that suppliers can embed re-usable core components as binary code into their ETCS target platforms. A detailed analysis is provided, indicating how future changes in the standard and project-specific adaptations, extensions and restrictions, can be accounted for in a new ETCS development, while minimizing the re-certification effort. It is shown for all phases of the development life cycle how the peer-reviewing capacity of the openETCS community may contribute to the correctness of the phases’ outputs, thereby increasing overall system dependability, with special emphasis on safety and security.}
\end{activity}

%##########################################################################
%conference papers and talks
%##########################################################################
\subsection{Conference papers and Talks}
\begin{activity}{FORMS/FORMAT 2012}
	\entry{Type}	{Conference Paper and Talk}
	\entry{Title}	{Using ERTMSFormalSpecs to model ERTMS braking curves}
	\entry{Authors}	{L. Ferier, S. Lukicheva and S. Pinte (ERTMS Solutions)}
	\entry{Date}		{11-13/12/2012}
	\entry{Location}	{Braunschweig, Germany}
	\website{http://www.forms-format.de}
	\desc{The European Railway Traffic Management System (ERTMS) defines standards for interoperability between the on-board train protection systems (ETCS) and the railway infrastructure. Part of this specification describes the computation of train braking curves and the associated train reactions according to its speed. This document explains how ERTMSFormalSpecs has been used to model such braking curves and the associated advantages of traceability and understandability of using a domain-specific language as opposed to a more common specification-implementation process, where multiple stages of human interpretation and interaction increase the opportunities for errors. }
\end{activity}

\begin{activity}{SEFM 2012, 10th International Conference on Software Engineering and Formal Methods}
	\entry{Type}	{Conference Paper and Talk}
	\entry{Title}	{Frama-C: a Software Analysis Perspective}
	\entry{Author}	{Pascal Cuoq, Florent Kirchner, Nikolay Kosmatov, Virgile Prevosto, Julien Signoles, and Boris Yakobowski}
	\entry{Date}		{01-05/10/2012}
	\entry{Location}	{Thessaloniki (Greece)}
%	\entry{Publisher}	{Springer (LNCS)}
	\website{http://sefm2012.city.academic.gr/}
	\desc{Frama-C is a source code analysis platform that aims at conducting verification of industrial-size C programs. It provides its users with a collection of plug-ins that perform static analysis, deductive verification, and testing, for safety- and security-critical software. Collaborative verification across cooperating plug-ins is enabled by their integration on top of a shared kernel and datastructures, and their compliance to a common specification language. This foundational article presents a synthetic view of the platform, its main and composite analyses, and some of its industrial achievements.}
\end{activity}

\begin{activity}{24th IFIP International Conference on Testing Software and Systems (ICTSS)}
	\entry{Type}	{Conference Paper and Talk}
	\entry{Title}	{Off-line test case generation for timed symbolic model-based conformance testing}
	\entry{Author}	{Christophe Gaston (CEA LIST)}
	\entry{Date}		{19-21/11/2012}
	\entry{Location}	{Aalborg (Denmark)}
%	\entry{Publisher}	{Springer (LNCS)}
	\website{http://ictss2012.aau.dk/}
	\desc{Model-based conformance testing of reactive systems consists in taking benefit from the model for mechanizing both test data generation and verdicts computation. On-line test case generation allows one to apply adaptive on-the-fly analyzes to generate the next inputs to be sent and to decide if observed outputs meet intended behaviors. On the other hand, in off-line approaches, test suites are pre-computed from the model and stored under a format that can be later performed on test-beds. In this paper, we propose a two-passes off-line approach where: for the submission part, a test suite is a simple timed sequence of numerical input data and waiting delays, and then, the timed sequence of output data is post-processed on the model to deliver a verdict. As our models are Timed Output Input Symbolic Transition Systems, our off-line algorithms involve symbolic execution and constraint solving techniques.}
\end{activity}

% Contribution not clear
%\begin{activity}{Workshop on Certification and Model-Driven Development of Safety Critical Software (ZeMoSS)}
%	\entry{Date}{27/02/2013}
%	\entry{Location}{Aachen, Germany}
%	\website{http://zemoss.in.tu-clausthal.de/}
%	%\entry{Interested Partners}{DLR}
%	\desc{The requirements for software quality of safety critical systems are very high, especially for the aspects of functional safety and information security. This German workshop covers among others the topics of model-driven development, integration of safety and security, and certification of open source software. Thus it is highly relevant for the project.}
%\end{activity}

\begin{activity}{FORMS/FORMATS 2014, 10th Symposium on Formal Methods}
	\entry{Type}	{Conference paper and Talk}
	\entry{Title}	{A Domain-specific Language for Railway Interlocking Systems}
	\entry{Author/Speaker}	{Linh H. Vu, Anne E. Haxtausen, Jan Peleska}
	\entry{Date}		{30/9-2/10/2014}
	\entry{Location}	{Brunswick}
	\entry{Website}		{\url{http://www.forms-format.de/}}
	\desc{This paper presents a domain-specific language (DSL) for
describing route-based interlocking systems which are compatible with
European Train Control System ETCS Level 2. The abstract syntax
and static semantics of the language are formally defined using the
RAISE Specification Language (RSL). Furthermore, the paper describes
an interlocking table generator (ITG) that generates automatically a
well-formed interlocking table from a well-formed railway network layout.
Experiments with the DSL and ITG using the RAISE tools and the C++
implementation show that the use of the DSL and ITG can increase
the productivity and significantly reduce errors in the specifications of
railway interlocking systems.}
\end{activity}

\begin{activity}{SPLC 2014, 18th Software Product Line Conference}
	\entry{Type}	{Conference paper and Talk}
	\entry{Title}	{A software product line approach for semantic specification of block libraries in dataflow languages}
	\entry{Author/Speaker}	{A. Dieumegard, A. Toom, M. Pantel}
	\entry{Date}		{15-19/9/2014}
	\entry{Location}	{Florence, Italy}
	\entry{Website}		{\url{http://http://www.splc2014.net/}}
	\desc{ missing information: issue9}
\end{activity}

% Contribution not clear
%\begin{activity}{ERTSS 2014, Embedded Real Time Software and Systems}
%	\entry{Type}	{Conference paper}
%	\entry{Title}	{A UML-MARTE temporal property verification tool based on model checking}
%	\entry{Author/Speaker}	{Ning Ge, Marc Pantel, Xavier Crégut}
%	\entry{Date}		{5-7/2/2014}
%	\entry{Location}	{Toulouse, France}
%	\entry{Website}		{\url{http://www.erts2014.org/}}
%	\desc{No abstract, Aknowledgement is not right, really openETCS publication}
%\end{activity}

\begin{activity}{SEFM 2013, 11th International Conference on Software Engineering and Formal Methods}
	\entry{Type}	{Conference paper and Talk}
	\entry{Title}	{Applied Bounded Model Checking for Interlocking System Designs}
	\entry{Author/Speaker}	{Anne Elisabeth Haxthausen, Jan Peleska, Ralf Pinger}
	\entry{Date}		{25-27/9/2013}
	\entry{Location}	{Madrid, Spain}
	\entry{Website}		{\url{http://antares.sip.ucm.es/sefm2013/}}
	\desc{In this paper the verification and validation of interlocking systems is investigated. Reviewing both geographical and route-related interlocking, the verification objectives can be structured from a perspective of computer science into (1) verification of static semantics, and (2) verification of behavioural (operational) semantics. The former checks that the plant model – that is, the software components reflecting the physical components of the interlocking system – has been set up in an adequate way. The latter investigates trains moving through the network, with the objective to uncover potential safety violations. From a formal methods perspective, these verification objectives can be approached by theorem proving, global, or bounded model checking. This paper explains the techniques for application of bounded model checking techniques, and discusses their advantages in comparison to the alternative approaches.}
\end{activity}

\begin{activity}{ICFEM 2014, 16th International Conference on Formal Engineering Methods}
	\entry{Type}	{Conference paper and Talk}
	\entry{Title}	{Complete Model-Based Equivalence Class Testing for the ETCS Ceiling Speed Monitor}
	\entry{Author/Speaker}	{Cécile Braunstein, Anne E. Haxtausen, Wen-ling Huang, Felix Hübner, Jan Peleska, Uwe Schulze, Linh Vu Hong}
	\entry{Date}		{3-7/11/2014}
	\entry{Location}	{Luxembourg City, Luxembourg}
	\entry{Website}		{\url{http://icfem2014.uni.lu/}}
	\desc{In this paper we present a new test model written in SysML
and an associated blackbox test suite for the Ceiling Speed Monitor
(CSM) of the European Train Control System (ETCS). The model is
publicly available and intended to serve as a novel benchmark for inves-
tigating new testing theories and comparing the capabilities of model-
based test automation tools. The CSM application inputs velocity val-
ues from a domain which could not be completely enumerated for test
purposes with reasonable effort. We therefore apply a novel method for
equivalence class testing that – despite the conceptually infinite cardi-
nality of the input domains – is capable to produce finite test suites
that are complete (i.e. sound and exhaustive) for a given fault model. In
this paper, an overview of the model and the equivalence class testing
strategy is given, and tool-based evaluation results are presented. For the
technical details we refer to the published model and a technical report
that is also available on the same website.}
\end{activity}

\begin{activity}{Rodin Workshop 2014, 5th Rodin User and Developer Workshop}
	\entry{Type}	{Conference Paper and Talk}
	\entry{Title}	{Event-B for Safety Analysis of Critical Systems}
	\entry{Author/Speaker}	{Matthias G\"{u}demann, Marielle Petit-Doche}
	\entry{Date}		{2-3/6/2014}
	\entry{Location}	{Toulouse, France}
	\entry{Website}		{\url{http://wiki.event-b.org/index.php/Rodin_Workshop_2014}}
	\desc{The Event-B modeling approach is designed to reason about the
correctness of systems in the early phases of the development
process. We propose its application to support safety activities of
a critical railway system and to strengthen the arguments for safety
cases for certification bodies. We provide insights into our usage
of Event-B for validation and verification of safety aspects using
the analysis of a reference model of the European Train Control
System (ETCS).}
\end{activity}

\begin{activity}{INDIN 2013, IEEE 11th International Conference on Industrial Informatics}
	\entry{Type}	{Conference Paper and Talk}
	\entry{Title}	{Formal Specification and Automated Verification of Railway Software with Frama-C}
	\entry{Author/Speaker}	{Jens Gerlach, Virgile Prevosto, Jochen Burghardt, Kerstin Hartig, Kim Völlinger, Hans Pohl}
	\entry{Date}		{29-31/7/2013}
	\entry{Location}	{Bochum, Germany}
	\entry{Website}		{\url{http://www.indin2013.org/n/}}
	\desc{This paper presents the use of the Frama-C toolkit for the formal verification of a model of train-controlling software against the requirements of the CENELEC norm EN 50128. We also compare our formal approach with traditional unit testing.}
\end{activity}

\begin{activity}{ERTSS 2014, Embedded Real Time Software and Systems}
	\entry{Type}	{Conference paper and Talk}
	\entry{Title}	{Formal specification of block libraries in dataflow languages}
	\entry{Author/Speaker}	{Arnaud Dieumegard, Andres Toom, Marc Pantel}
	\entry{Date}		{5-7/2/2014}
	\entry{Location}	{Toulouse, France}
	\entry{Website}		{\url{http://www.erts2014.org/}}
	\desc{Graphical dataflow-style modeling languages like Simulink and Scicos are widely used in the development of embedded control systems as high-level engineering languages. A signifcant part of their modeling power is captured in function block libraries. In this paper we present an on-going work on the model-based formalisation of such libraries, which intends to bridge the gaps between the different parts of the development process: high-level requirements, design, implementation and verification. Our approach is based on a specification domain specific language (DSL), which captures the variability of blocks through a software product line approach. We have defined translations to other languages like the WHY3 language for verification, different documentation formats, code generator configuration files, etc. These experiments have been carried out in the context of the GeneAuto embedded code generator project and are being extended and applied in its successor projects ProjectP and Hi-MoCo.}
\end{activity}

\begin{activity}{SEKE 2014, The 26th International Conference on Software Engineering and Knowledge Engineering}
	\entry{Type}	{Conference paper and Talk}
	\entry{Title}	{Formal Verification of Coordination Systems' Requirements - A Case Study on the European Train Control System}
	\entry{Author/Speaker}	{Huu Nghia Nguyen, Ana Cavalli}
	\entry{Date}		{1-3/7/2014}
	\entry{Location}	{Hyatt Regency, Vancouver, Canada}
	\entry{Website}		{\url{http://www.ksi.edu/seke/seke14.html}}
	\desc{Formal verification techniques of system requirements such as model-checking and theorem proving aims to show that the requirements satisfy some properties. Consequently, their success depends on the quality of the properties formulation. We propose an approach to verify requirements of coordination systems by generating automatically the properties to be verified from the requirements themselves. The requirement specifications of a system are provided at two different levels. The coordination specification gives a global overview of the system, in terms of the different roles participating to it, with their goals and needs and with their mutual dependencies and expectations. The process specification shows how a local participant of the system performs its activities. We exploit model checking techniques for verifying the process requirements against the properties generated by the coordination requirements. In addition to provide a theoretical framework, we show how to apply this methodology on the verification of the System Requirement Specification of the European Train Control System. It is complemented with a toolchain.}
\end{activity}

\begin{activity}{INDIN 2013, IEEE 11th International Conference on Industrial Informatics}
	\entry{Type}	{Conference paper and Talk}
	\entry{Title}	{Graphical Modelling meets Formal Methods}
	\entry{Author/Speaker}	{Stefan Gulan, Sven Johr, Roberto Kretschmer, Stefan Rieger, Michael Ditze}
	\entry{Date}		{29-31/7/2013}
	\entry{Location}	{Bochum, Germany}
	\entry{Website}		{\url{http://www.indin2013.org/n/}}
	\desc{The graphical modelling languages UML and SysML, nowadays widely used in industry, integrate different modelling concepts and notations in one standardised framework.
However, they lack a clearly defined, unambiguous semantics and thus their formal verification represents a challenge. On the other hand, current safety standards,
including ISO 26262, demand such verification especially for safety-relevant systems. The literature proposes a plethora of different semantics and formalisms for
UML/SysML. In this paper we compare and  summarise existing work on the formalisation of behavioural UML and SysML models and their verification. Our goal is to foster
a better understanding of the problems related to UML/SysML formalisation, and to aid people bridging the gap from high level graphical modelling to formal verification
techniques.}
\end{activity}

\begin{activity}{ACATTA 2013, 1st IFAC Workshop on Advances in Control and Automation Theory for Transportation Applications}
	\entry{Type}	{Conference paper and Talk}
	\entry{Title}	{Integration of Petri Nets into STAMP/CAST on the example of Wenzhou 7.23 accident.}
	\entry{Author/Speaker}	{Dirk Spiegel, René Sebastian Hosse, Jan Welte, Eckehard Schnieder}
	\entry{Date}		{16-17/9/2013}
	\entry{Location}	{Istanbul, Turkey}
	\entry{Website}		{\url{http://www.acatta13.itu.edu.tr/}}
	\desc{This paper illustrates how formal models, in this case Petri nets, can be integrated into the promising STAMP approach. CENELEC 5012X recommend applying formal methods to demonstrate safety in the railway sector. STAMP does not include formal modeling and therefore seems inadequate for safety analysis in railway traffic. This drawback can be eliminated by hybridizing the approach with the ProFunD hazard analysis (DIN EN 62551) using Petri net models. Such a new method, called formalSTAMP that successfully integrates a formal model into STAMP is introduced in this paper. A sample accident analysis application on the Wenzhou 7.23 accident in China is performed and the results are presented in contrast to an original STAMP / CAST analysis by (Dong 2012).}
\end{activity}

\begin{activity}{FORMS/FORMAT 2012, 9th Symposium on Formal Methods}
	\entry{Type}	{Conference paper and Talk}
	\entry{Title}	{Model Based Development and Tests for openETCS Applications – A Comprehensive Tool Chain}
	\entry{Author/Speaker}	{Johannes Feuser, Jan Peleska}
	\entry{Date}		{12-13/12/2012}
	\entry{Location}	{Brunswick, Germany}
	\entry{Website}		{\url{http://www.forms-format.de/2012/}}
	\desc{This paper presents research results on model-based development and testing for the European Train Control System (ETCS). We focus on the tool chain developed by the authors, which supports the creation of graphical formal specifications of the ETCS System Requirement Specification, code generation, verification and validation.}
\end{activity}

\begin{activity}{ICSSEA 2013, 25th International Conference on Software and Systems Engineering and their Applications}
	\entry{Type}	{Conference paper and Talk}
	\entry{Title}	{Model-Based Testing and Test Case Generation in the Context of the Open ETCS Project}
	\entry{Author/Speaker}	{Cyril Cornu, Christophe Gaston, Agnès Lanusse, Frédérique Vallée}
	\entry{Date}		{4-6/11/2013}
	\entry{Location}	{Paris, France}
	\entry{Website}		{\url{http://icssea.enst.fr/icssea13/}}
	\desc{missing input}
\end{activity}

\begin{activity}{4th International ABZ Conference}
	\entry{Type}	{Conference paper and Talk}
	\entry{Title}	{Model-Checking Real-Time Properties of an Aircraft Landing Gear System Using Fiacre}
	\entry{Author/Speaker}	{Silvano Dal Zilio, Lukasz Fronc, Bernard Berthomieu}
	\entry{Date}		{1-6/6/2014}
	\entry{Location}	{Toulouse, France}
	\entry{Website}		{\url{http://hal.archives-ouvertes.fr/hal-00967422}}
	\desc{We describe our experience with modeling the landing gear system of an aircraft using the formal specification language Fiacre. Our model takes into account the behavior and timing properties of both the physical parts and the control software of this system. We use this formal model to check safety and real-time properties on the system but also to find a safe bound on the maximal time needed for all gears to be down and locked (assuming the absence of failures). Our approach ultimately relies on the model-checking tool Tina, that provides state-space generation and model-checking algorithms for an extension of Time Petri Nets with data and priorities.}
\end{activity}

\begin{activity}{IPCT, The International Conference on Advances in Information Processing and Communication Technology}
	\entry{Type}	{Conference paper and Talk}
	\entry{Title}	{On Modeling and Testing Components of the European Train Control System}
	\entry{Author/Speaker}	{César Andrés, Ana Cavalli, Nina Yevtushenko, João Santos, Rui Abreu}
	\entry{Date}		{7-8/6/2014}
	\entry{Location}	{Rome, Italy}
	\entry{Website}		{\url{http://ipct.theired.org/}}
	\desc{This paper studies the abilities of the formal model of a Timed Extended Finite State Machine (TEFSM) to represent the safety properties of the European Train Control System (ETCS). The model is based on Finite State Machines augmented with continuous variables and time information, which allows representing the basic functioning of the units in this real-time system. In order to represent temporal requirements, timeouts are used for modeling some aspects of the (internal) critical behavior of the train control system. The model abilities to represent safety properties are evaluated using different testing scenarios for model implementations in IF, XML and JAVA languages. Tests are automatically generated using the tool TestGen-IF where corresponding safety properties are specified as test objectives. Based on the obtained experimental results the advantages and disadvantages of a developed model are briefly discussed.}
\end{activity}

\begin{activity}{ERTSS 2014, Embedded Real Time Software and Systems}
	\entry{Type}	{Conference paper and Talk}
	\entry{Title}	{Probabilistic failure analysis in model Validation and Verification}
	\entry{Author/Speaker}	{Ning Ge, Marc Pantel, Xavier Crégut}
	\entry{Date}		{5-7/2/2014}
	\entry{Location}	{Toulouse, France}
	\entry{Website}		{\url{http://www.erts2014.org/}}
	\desc{missing input}
\end{activity}

\begin{activity}{ERTSS 2014, Embedded Real Time Software and Systems}
	\entry{Type}	{Conference paper and Talk}
	\entry{Title}	{Rail, Space, Security: Three Case Studies for SPARK 2014}
	\entry{Author/Speaker}	{Claire Dross, Pavlos Efstathopoulos, David Lesens, David Mentré, Yannick Moy}
	\entry{Date}		{5-7/2/2014}
	\entry{Location}	{Toulouse, France}
	\entry{Website}		{\url{http://www.erts2014.org/}}
	\desc{SPARK is a subset of the Ada programming language targeted at safety- and security-critical applications. SPARK 2014 is a major evolution of the SPARK language and toolset, that integrates formal program verification into existing development processes, in order to decrease the cost of software verication, subject to certification constraints. We present industrial case studies in three different certification domains that show the benefits of using formal verification with SPARK 2014.}
\end{activity}

\begin{activity}{PNSE 2013, International Workshop on Petri Nets and Software Engineering}
	\entry{Type}	{Conference paper and Talk}
	\entry{Title}	{Real-Time Property Specific Reduction for Time Petri Net}
	\entry{Author/Speaker}	{Ning Ge, Marc Pantel}
	\entry{Date}		{23-24/6/2014}
	\entry{Location}	{Tunis, Tunisia}
	\entry{Website}		{\url{http://www.informatik.uni-hamburg.de/TGI/events/pnse14/}}
	\desc{This paper presents a real-time property specific reduction approach for Time Petri Net (TPN). It divides TPN models into sub-nets of smaller size, and constructs an abstraction of reducible ones, which exhibits the same property specific behavior, but has less transitions and states. This directly reduces the amount of computation needed to generate the whole state space. This method adapts well to the verification of real-time properties in asynchronous systems. It should be possible to apply similar methods to other families of properties.}
\end{activity}

\begin{activity}{WCRR2013, 10th World Congress on Railway Research}
	\entry{Type}	{Conference paper and Talk}
	\entry{Title}	{Survey of formal model-based development of safety-critical software for railway applications}
	\entry{Author/Speaker}	{Jan Welte, Hansjörg Manz, Eckehard Schnieder}
	\entry{Date}		{25-28/11/2013}
	\entry{Location}	{Sydney, Australia}
	\entry{Website}		{\url{http://www.wcrr2013.org/}}
	\desc{The OpenETCS project has the goal to develop an integrated approach for development and implementation of software of European Train Control System (ETCS) on-board units. Thereby, the OpenETCS concept is based on the use of methods and tools which support the formal specification and verification of requirements in an overall model-based development process. To provide transparency and allow compatibility over the life cycle of the train system “Open Standards” shall be utilized on all levels. This paper presents an overview on existing methods used in the railway sector and other comparable industries for software development including verification and validation. Therefore a number of interviews with experts from different organisations and various fields of expertise have been conducted to learn about their approaches and experience. In addition the possibility of integrating “Open Standards” in the existing development process has been discussed. Based on these interviews requirements have been derived, which have to be addressed during the further work of the OpenETCS project.}
\end{activity}

\begin{activity}{Formats 2014, 12th International Conference on Formal Modeling and Analysis of Timed Systems}
	\entry{Type}	{Conference paper and Talk}
	\entry{Title}	{Time Petri Nets with Dynamic Firing Dates: Semantics and Applications}
	\entry{Author/Speaker}	{Silvano Dal Zilio, Lukasz Fronc, François Vernadat}
	\entry{Date}		{8-10/9/2014}
	\entry{Location}	{Florence, Italy}
	\entry{Website}		{\url{http://formats2014.unifi.it/}}
	\desc{We define an extension of time Petri nets such that the time at which a transition can fire, also called its firing date, may be dynamically updated. Our extension provides two mechanisms for updating the timing constraints of a net. First, we propose to change the static time interval of a transition each time it is newly enabled; in this case the new time interval is given as a function of the current marking. Next, we allow to update the firing date of a transition when it is persistent, that is when a concurrent transition fires. We show how to carry the widely used state class abstraction to this new kind of time Petri nets and define a class of nets for which the abstraction is exact. We show the usefulness of our approach with two applications: first for scheduling preemptive task, as a poor man's substitute for stopwatch, then to model hybrid systems with non trivial continuous behavior.}
\end{activity}

\begin{activity}{IPCT, The International Conference on Advances in Information Processing and Communication Technology}
	\entry{Type}	{Conference paper and Talk}
	\entry{Title}	{Verifying and testing ETCS Train Implementations based on IF specifications}
	\entry{Author/Speaker}	{Natalia Kushik, Denisa Ianculescu, Ana Cavalli, Mounir Lallali}
	\entry{Date}		{7-8/6/2014}
	\entry{Location}	{Rome, Italy}
	\entry{Website}		{\url{http://ipct.theired.org/}}
	\desc{This paper presents test generation scenarios for a train implementation based on the requirements for European Train Control System (ETCS). The formal model used for the test derivation is the model of a Timed Extended Finite State Machine (TEFSM) given in the IF language. This language allows to capture some important properties such as safety properties that should be checked for train implementations represented as corresponding test objectives. The tool TestGen-IF is then used for automatic generation of test cases.}
\end{activity}

\begin{activity}{INFORMATIK 2014, Workshop: Technologien zur Analyse und Steuerung komplexer cyber-physischer Systeme (CPSData)}
	\entry{Type}	{Conference paper and Talk}
	\entry{Title}	{Ein abstraktes SystemC-Modell zur Analyse und Leistungsabschätzung des europäischen Zugsicherungssystems ETCS}
	\entry{Author/Speaker}	{Benjamin Beichler, Alexander Nitsch, Frank Golatowski, Christian Haubelt}
	\entry{Date}		{22-26/9/2014}
	\entry{Location}	{Stuttgart, Germany}
	\entry{Website}		{\url{http://www.informatik2014.de/workshops14.html}}
	\desc{In diesem Beitrag wird ein SystemC-Modell der Geschwindigkeits- und Abstandsüberwachung aus dem European Train Control System (ETCS) vorgestellt. Dieses Modell dient als Ausgangspunkt für die frühzeitige Abschätzung der Leistungsfähigkeit des Systems und die in der Berechnung entstehenden Datenmengen. Hierfür wurde eine neuartige Methode entwickelt, welche es bei minimalen Anpassungen am Anwendungsmodell erlaubt, schnell unterschiedliche Entwurfsalternativen zu explorieren. Dabei werden SystemC-Prozesse in Abhängigkeit von Scheduling-Entscheidungen gestartet, deren Kommunikationsverhalten aufgezeichnet und anschließend Ausführungs- und Kommunikationszeiten simuliert.}
\end{activity}

\subsection{Book chapter}

\begin{activity}{Railway Safety, Reliability, and Security: Technologies and Systems Engineering}
	\entry{Type}	{Book chapter}
	\entry{Title}	{The Model-Driven openETCS Paradigm for Secure, Safe and Certifiable Train Control Systems}
	\entry{Author/Speaker}	{Jan Peleska, Johannes Feuser, Anne Elisabeth Haxthausen}
	\entry{Date}		{5/2012}
	%\entry{Location}	{}
	\entry{Website}		{\url{http://www.igi-global.com/chapter/model-driven-openetcs-paradigm-secure/66666}}
	\desc{A novel approach to managing development, verification, and validation artifacts for the European Train Control System as open, publicly available items is analyzed and discussed with respect to its implications on system safety, security, and certifiability. After introducing this so-called model-driven openETCS approach, a threat analysis is performed, identifying both safety and security hazards that may be common to all model-based development paradigms for safety-critical railway control systems, or specific to the openETCS approach. In the subsequent sections state-of-the-art methods suitable to counter these threats are reviewed, and novel promising research results are described. These research results comprise domain-specific modeling, model-based code generation in combination with automated object code verification and explicit utilization of virtual machines to ensure containment of security hazards.}
\end{activity}

%##########################################################################
%internally Publications
%##########################################################################

\subsection{Project internally publications}

\begin{activity}{OpenETCS WP4 Contribution}
	\entry{Type}	{Technical report}
	\entry{Title}	{A SysML Test Model and Test Suite for the ETCS Ceiling Speed Monitor}
	\entry{Author/Speaker}	{Jan Peleska, Cecile Braunstein, Uwe Schulze, Felix Hübner, Wen-ling Huang, Anne E. Haxthausen, Linh Vu Hong}
	\entry{Date}		{11/5/2014}
	\entry{Location}	{Bremen/Hamburg, Germany}
	\entry{Website}		{\url{http://www.informatik.uni-bremen.de/agbs/testingbenchmarks/openETCS/ceiling-speed-monitoring/testing_the_etcs_csm.pdf}}
	\desc{}
\end{activity}

%##########################################################################
%talks
%##########################################################################

\subsection{Talks}

\begin{activity}{4th annual Signalling and Train and Control Conference}
	\entry{Type}	{Talk}
	\entry{Title}	{Infrastructure manager case study: Ensuring systems integration and developing functionality with openETCS}
	\entry{Speaker}	{Klaus-R\"{u}diger Hase (DB)}
	\entry{Date}		{19-21/03/2013}
	\entry{Location}	{Vienna, Austria}
	\entry{Website}		{\url{http://globaltransportforum.com/signalling-and-train-control/}}
	\desc{The 4th annual Signalling and Train Control show in Vienna is one of the definitive events for rail signalling, telecom and traffic management experts. With over 50 leading speakers and 300 attendees, the congress allows for the sharing of best practice strategies and unparalleled business development opportunities. At the symposium openETCS project leader Klaus-Rüdiger Hase presented the openETCS approach to developing functionality and system integration.}
\end{activity}

%missing in Zotro
\begin{activity}{12th International SIGNAL+DRAHT-Congress 2012}
	\entry{Type}	{Talk}
	\entry{Title}	{openETCS: Von der Idee zur Praxis}
	\entry{Speaker}	{Klaus-R\"{u}diger Hase (DB)}
	\entry{Date}		{08-09/11/2012}
	\entry{Location}	{Fulda, Germany}
	\desc{The openETCS project has been presented at the congress, which took place under the motto ``How will signaling technology evolve over the next ten to fifteen years?''. The focus was predominantly on the development of the railway networks as well as on the expected changes in the areas of automatic
train control, route protection, level crossing protection and control technology. Therefore, this congress was an ideal platform to present the project to a large audience of highly distinguished experts.}
\end{activity}

\begin{activity}{APTA/UIC High-Speed Congress 2012}
	\entry{Type}	{Talk}
	\entry{Title}	{openETCS: Applying Open Proof's to the European Train Control System}
	\entry{Speaker}	{Klaus-R\"{u}diger Hase (DB)}
	\entry{Date}		{10-13/07/2012}
	\entry{Location}	{Philadelphia, PA, USA}
	\desc{The Congress focuses on highspeed railway traffic. Here, the openETCS project has been presented a large audience of highly distinguished experts and has especially been presented to the non-european audience.}
\end{activity}

\begin{activity}{41th Symposium "Moderne Schienenfahrzeuge"}
	\entry{Type}	{Talk}
	\entry{Title}	{Open ETCS: Ein internationales ITEA2-Projekt begleitet den Wandel}
	\entry{Speaker}	{Klaus-R\"{u}diger Hase (DB)}
	\entry{Date}		{7-10/4/2013}
	\entry{Location}	{Graz, Austria}
	\desc{The symposium focuses on highspeed railway traffic, railway transport vehicles, rail fraight transport, vehicle components, and interaction of wheel and track. The platform of the symposium will be used to present the openETCS approach to a large audience of distinguished experts in railway business.}
\end{activity}

\begin{activity}{iFM 2013: 10th International Conference on integrated Formal Methods}
	\entry{Type}	{Talk}
	\entry{Title}	{Tutorial: Specification and Proof of Programs with Frama-C}
	\entry{Speaker}	{Virgile Prevosto}
	\entry{Date}		{10-14/6/2013}
	\entry{Location}	{Turku, Finland}
	\desc{Despite the spectacular progress made by the developers of formal verification tools, their usage remains basically reserved for the most critical software. More and more engineers and researchers today are interested in such tools in order to integrate them into their everyday work. This half-day tutorial proposes a practical introduction during which the participants will write C program specifications, observe the proof results, analyze proof failures and fix them. Participants will be taught how to write a specification for a C program, in the form of function contracts, and easily prove it with an automatic tool in FRAMA-C, a freely available software verification toolset. Those who will have FRAMA-C and JESSIE installed (e.g. from ready-to-install packages frama-c, why, alt-ergo under Linux) will also run automatic proof of programs on their computer. Program specifications will be written in the specification language ACSL similar to other specification languages like JML that some participants may know. ACSL-syntax is intentionally close to C and can be easily learned on-the-fly.}
\end{activity}

\begin{activity}{Symposium Test4Rail}
	\entry{Type}	{Talk}
	\entry{Title}	{Early Verification of Concepts on the Example of openETCS}
	\entry{Speaker}	{Marc Behrens}
	\entry{Date}		{29-30/10/2013}
	\entry{Location}	{Brunswick, Germany}
	\desc{The concept of early verification is one key factor in shortening the development cycle by parallelizing development and verification. With its integrated model based approach and proof of transformation tests within openETCS tests are shifted to model level transforming the general V- Model to a Y-Model. The symposium focused on new approaches to verification of safety critical software in the railway sector. The concept of early verification has been presented to an audience of highly distinguished international experts.}
\end{activity}